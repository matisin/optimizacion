\documentclass[12pt]{article}
\usepackage[utf8]{inputenc}
\usepackage{graphicx}
\usepackage{framed, color}
\usepackage{paralist, blindtext}

\definecolor{shadecolor}{rgb}{0.83,1,0.8}
\begin{document}

\begin{titlepage}

\newcommand{\HRule}{\rule{\linewidth}{0.5mm}} % Defines a new command for the horizontal lines, change thickness here

\center % Center everything on the page
 
%----------------------------------------------------------------------------------------
%	HEADING SECTIONS
%----------------------------------------------------------------------------------------

\textsc{\LARGE Universidad de Concepción}\\[1.5cm] % Name of your university/college
\textsc{\Large Departamento de Ingeniería Civil Informática y Ciencias de la Computación}\\[0.5cm] % Major heading such as course name
\textsc{\large Optimización}\\[0.5cm] % Minor heading such as course title

%----------------------------------------------------------------------------------------
%	TITLE SECTION
%----------------------------------------------------------------------------------------

\HRule \\[0.4cm]
{ \huge \bfseries Hub location problem }\\[0.4cm] % Title of your document
\HRule \\[1.5cm]
 
%----------------------------------------------------------------------------------------
%	AUTHOR SECTION
%----------------------------------------------------------------------------------------

\begin{minipage}{0.4\textwidth}
\begin{flushleft} \large
\emph{Author:}\\
Cristobal \textsc{Donoso}
Matías \textsc{Medina}
\end{flushleft}
\end{minipage}
~
\begin{minipage}{0.4\textwidth}
\begin{flushright} \large
\emph{Profesora:} \\
Lorena \textsc{Pradenas} % Supervisor's Name
\end{flushright}
\end{minipage}\\[4cm]

{\large \today}\\[3cm] % Date, change the \today to a set date if you want to be precise
\vfill % Fill the rest of the page with whitespace
\end{titlepage}

%----------------------------------------------------------------------------------------
%	INTRODUCTION
%----------------------------------------------------------------------------------------
\section{Descripción del problema}
El problema consiste en satisfacer demandas de información, gente o materias primas entre nodos origen y destino. Los hubs son ubicados en posiciones que reducen las conexiones entre nodos, por ejemplo, una red completa de k nodos tendría $k(k-1)$ enlaces origen-destino. Sin embargo, si un hub es seleccionado para conectar todos los nodos origen-destino, entonces se tendría tan solo $2(k-1)$ conexiones para suplir toda la demanda. Este problema se pude extender a más de un nodo-hub en la red, llamada red de multiples hub.


\end{document}